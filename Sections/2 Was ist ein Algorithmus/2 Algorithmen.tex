\documentclass[
../../AuD-Zusammenfassung.tex,
]
{subfiles}

\externaldocument[ext:]{../../AuD-Zusammenfassung}
% Set Graphics Path, so pictures load correctly
\graphicspath{{../../}}

\begin{document}
\section{Was ist ein Algorithmus?}
Ein Algorithmus beschreibt eine Handlungsvorschrift zur Umwandlung von Eingaben in eine Ausgabe.\\
Dabei sollte ein Algorithmus im allgemeinen folgende Vorraussetzungen erfüllen:
\begin{enumerate}
    \item Bestimmt:
    \begin{itemize}
        \item Determiniert: Bei gleicher Eingabe liefert der Algortihmus gleiche Ausgabe. \\$\implies$ Ausgabe nur von Eingabe abhängig, keine äußeren Faktoren.
        \item Determinismus: Bei gleicher Eingabe läuft der Algorithmus immer gleich durch die Eingabe. \\$\implies$ Gleiche Schritte, Gleiche Zwischenstände.
    \end{itemize}
    \item Berechenbar:
    \begin{itemize}
        \item Finit: Der Algorithmus ist als endlich definiert. (Theoretisch)
        \item Terminierbar: Der Algorithmus stoppt in endlicher Zeit. (Praktisch)
        \item Effektiv: Der Algorithmus ist auf Maschine ausführbar.
    \end{itemize}
    \item Andwendbar:
    \begin{itemize}
        \item Allgemein: Der Algorithmus ist für alle Eingaben einer Klasse anwendbar, nicht nur für speziellen Fall.
        \item Korrekt: Wenn der Algorithmus ohne Fehler terminiert, ist die Ausgabe korrekt.
    \end{itemize}
\end{enumerate}

\end{document}